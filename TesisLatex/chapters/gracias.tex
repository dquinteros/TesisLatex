
\begin{gracias}

Este es quizá el espacio en blanco mas difícil de llenar de todo este trabajo, no por que no haya nada que agradecer si no por que siempre faltará espacio.

Agradezco por supuesto a mi familia por su apoyo constante desde siempre. A los mas cercanos mamá, papá y hermana, que siempre están ahí para tenderme una mano, pero también aquellos mas lejanos que con su cariño hicieron de este proceso algo mucho mas ameno.

También es importante agradecer a mis profesores partiendo desde aquellos que en el colegio me motivaron desde sus distintos puntos de vista para seguir esta carrera y luego a quienes les sucedieron en la universidad y me permitieron observar el mundo muchas formas distintas, finalmente al guía de este trabajo el profesor Sergio Velastin un soporte importante para la investigación y un mentor que proyecta una sencillez única. 

Hay también quienes por sus diversos aportes deben ser mencionados por su nombre y apellido, pero yo nunca les llamo así ojalá no se me quede ninguno. El Weru, quizá sin su ayuda esto aún no estaría listo; el Gera, un amigo de muchas y muchas más; el Aldo y la Gaby, los pongo juntos por el espacio pero cada uno tiene lo suyo; el Armando, un muy buen y querido amigo; a mi tío Johny, quien me ayudo con su inmenso cariño y conocimiento en la corrección de este trabajo; a la Olga y la Pauli, por sus ayudas en este trabajo.

Finalmente un agradecimiento muy amoroso a Karín Acevedo, por tu apoyo, tu cariño, por lo que vivimos y viviremos.

El trabajo que se describe aquí se llevó a cabo como parte del proyecto OBSERVE financiado por el programa Regular Fondecyt de Conicyt en virtud de concesión no. 1140209.
\paginaenblanco
\end{gracias}