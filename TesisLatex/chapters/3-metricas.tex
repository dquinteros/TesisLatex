%--------Métricas
%--------Daniel Quinteros Céspedes
%--------23-09-2014

\chapter{M\'etricas de sensibilidad espacial}
\label{cap:metricas}


\section{Sensibilidad espacial}
\label{metricas:sensibilidad}

La sensibilidad es según la real academia española 

\section{M\'etricas propuestas}
\label{metricas:propuestas}

Las m\'etricas propuestas para describir la sensibilidad espacial son la curtosis (\textit{kurtosis}) y la asimetría (\textit{skewness}) las cuales representan 

\subsection{Evaluaci\'on teórica de las m\'etricas propuestas}
\label{propuestas:evaluacion}

Una vez propuestas la m\'etricas es necesario evaluar su contingencia respecto del problema de la sensibilidad espacial, con este fin se desarroll\'o un software ejemplo preliminar que permite obtener un matriz resultado con valores correspondientes al grado de seguridad con la que un clasificador indica la existencia de un peatón alrededor del \textit{ground truth}, para  este se utiliz\'o la implementación del detector de objetos por Histograma de gradientes orientados \citep{Dalal2005} el cual viene incluido por defecto dentro de la suite de librer\'ias de OpenCV, adem\'as se consider\'o como \textit{ground truth} el punto indicado en la anotaciones realizadas por Dalal que acompa\~nan la descarga del set de datos INRIA

En primer lugar se obtiene eñ 



\section{Conclusiones}
\label{metricas:conclusiones}