\chapter{Glosario}
\label{cap:glosario}

En este anexo se presenta la terminología relacionada al problema desarrollado en este trabajo de título.

\begin{itemize}

\item Adaboost: Algoritmo de aprendizaje basado en la idea del impulso (boosting), esto es, crear una regla de predicción muy precisa a partir de la combinación de varias reglas débiles y poco precisas.

\item Asimetría (estadística): medida que indica la simetría de la distribución de una variable respecto a la media aritmética, indicando de esta forma si existe el mismo número de elementos a la izquierda y a la derecha de la media de tal distribución. 

\item Clasificador: corresponde a un mapeo de instancias no etiquetadas a clases, posiblemente discretas. Los clasificadores poseen una forma y un procedimiento de interpretación.

\item Curtosis: Grado de concentración que presentan los valores alrededor de la zona central de distribución.Grado de agudeza o achatamiento de una distribución con relación a la distribución normal, es decir, mide cuán puntiaguda es una distribución.

\item Gradiente: (En análisis espacial) Variación de intensidad de un fenómeno por unidad de distancia entre un lugar y un centro o eje dado. Operación vectorial que opera sobre una función escalar, cuyo producto es un vector cuya magnitud es la máxima razón de cambio de la función en el punto del gradiente y que apunta en la dirección de ese máximo.

\item Ground truth: conjunto de mediciones que se sabe que son más precisas que las mediciones que el sistema que se está poniendo a prueba. Determina que tan preciso es el modelo que se está probando.

\item Métrica: Medida cuantitativa del grado en que un sistema, componente o proceso posee un atributo dado. Distancia asociada a un espacio métrico, que cumple con los cuatro axiomas de métrica, positividad, desigualdad triangular, identidad de los indiscernibles, simetría. 

\item Set de datos: Colección de información que posee un nombre, que contiene unidades de información organizadas de una manera específica. Un esquema (descripción de los atributos de un set de datos y sus atributos) y un conjunto de instancias que cumplen con el esquema.

\item Support Vector Machine (SVM): Tipo de clasificador discriminatorio definido formalmente por un hiperplano separador. Corresponde a un tipo de aprendizaje supervisado.

\item Visión por computador: Corresponde a una rama de la Inteligencia Artificial que tiene por objetivo modelar matemáticamente los procesos de percepción visual de los seres vivos y generar programas que permitan simular estas capacidades visuales por computador.

\end{itemize}
