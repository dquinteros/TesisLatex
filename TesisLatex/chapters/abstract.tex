

\resumenCastellano{

El problema de la detección de personas para el conjunto de datos INRIA es presentado como resuelto en un trabajo anterior por \cite{dalal2006}. Sin embargo, aún existen muchos sub problemas sin resolver. Uno de estos problemas es la caracterización de la ``sensibilidad'' espacial de un detector de personas, es decir que tan bien sintonizada es la detección, especialmente dentro un pequeña vecindad alrededor del peatón a ser detectado. Esta propiedad, permite simplificar el método \textit{non-maximal suppression} que normalmente es utilizado para eliminar detecciones múltiples. 
Este trabajo de titulo estudia este problema en detalle, considerando una vecindad alrededor de cada peatón anotado en el conjunto de verdad, evaluando la respuesta de un clasificador y transformando esta respuesta en una probabilidad (ya que algunos clasificadores tienen por salida una distancia sin interpretación probabilística directa), lo que permite realizar comparaciones entre diferentes clasificadores y mejorar la visualización de los resultados.

\vspace*{0.5cm}
\KeywordsES{Sensibilidad espacial, HOG, SVM, Adaboost, Imágenes, Detección de peatones}
}

\newpage

\resumenIngles{

The people detection problem for the INRIA pedestrian data set is presented as solved in earlier work by \cite{dalal2006}, but still has many unsolved sub problems. One of these problems is the characterisation of the spatial ``sensitivity'' of the people detector \ie how sharp its detection is especially within a small neighbourhood around the pedestrians to be detected, a property that simplifies the non-maximal suppression method that is normally needed to eliminate multiple detections. 
This work studies this problem in some detail by taking a neighbourhood around the ground truth and evaluating a classifier's response by converting such response into a probability (as some classifiers output a distance without a direct probabilistic interpretation) to allow comparisons between different classifiers and improve the results visualization.

\vspace*{0.5cm}
\KeywordsEN{Spatial Sensitivity, HOG, SVM, Adaboost, Images, Pedestrian detection}

}